\documentclass[12pt]{niuthesis}
% \documentclass[12pt,singlespacing]{niuthesis}

% some packages are loaded here, include any others you need here
\usepackage{latexsym}		% to get LASY symbols
\usepackage{graphicx}		% to insert PostScript figures
\usepackage{rotating}           % defines sidways table and figure env.
% \usepackage{hyperref}		% to insert hyperrefs
\usepackage{natbib}    % for better bibliography management

%%%%%%%%%%%%%%%%%%%%%%%%%%%%%%%%%%%%%%%%%%%%%%%%%%%%%%%%%%%%%%%%
% Some LaTeX macros
%%Here is where you can define any custom LaTeX macros or commands you want to use in your dissertation.


%%%%%%%%%%%%%%%%%%%%%%%%%%%%%%%%%%%%%%%%%%%%%%%%%%%%%%%%%%%%%%%%%%
% Choose the chapter(s) / files you want to work with

%%% use this to compile all chapters
\def\files{Chapter1/ch1,refs}

%%% use this to work with only one chapter
% \def\files{ch2}

\includeonly{\files}

%%%%%%%%%%%%%%%%%%%%%%%%%%%%%%%%%%%%%%%%%%%%%%%%%%%%%%%%%%%%%%%%%
\begin{document}

\title{Your Dissertation Title Here}

\author{Your Name}

\major{Your Major/Program}
\degree{Dissertation}{Ph.D.}{Doctor of Philosophy}
\degreedate{May}{Year}
\department{Department of XXXX}
\director{Your Advisor's Name}

\begin{abstract}
  This is where you write your abstract of the dissertation as a whole. The abstract should briefly introduce the problem, summarize the chapters, and preview some conclusions. A diss abstract can be longer than a typical journal article abstract, perhaps even a page or two.
\end{abstract}

\begin{acknowledgments}
  Here's where you acknowledge folks who helped, probably your advisor and committee, perhaps funding agencies, influential professors and peers and anyone else you want to thank. You can be as formal or informal as you like here.
\end{acknowledgments}

\begin{dedication}
  Here is where you can dedicate your dissertation to someone, if you wish. Common dedications include spouses, family members, mentors, or anyone else who has been particularly supportive. You can be as formal or informal as you like here, and you can even include a quote if you like. This is not required, but it is a moment to be very sweet and reflect on the PhD journey as a whole.
\end{dedication}

% comment this to suppress prologue
\MakeThesisPrologue % includes table of contents
% \tableofcontents

\chapter{Introduction}		% chapter 1
\label{introchap}		% for reference (\ref{introchap})

Words go here. This example is an adaptation of Roland Winkler's example thesis \citep{winkler2015niuthesis}.

The natbib package allows differentiation between parenthetical citations (like the example above) or in-text citations. You may find \citet{booth2008craft} to be useful reading in framing your research and writing your dissertation.

%%Since this is a document you are inputting, you do not need \begin{document} or \end{document}.  You also do not need to include the preamble (the stuff before \begin{document} in the main file).  You can just start with the content of the chapter.  You can use \section, \subsection, etc as needed.  

%%You can also include figures, tables, equations, etc. If you are trying to organize your files by putting them in the relevant chapter folder, make sure to include their directory correctly. (i.e. \includegraphics{Chapter1/figure1} if the figure is in the Chapter1 folder).  You can also put figures in the main folder or separate folder if you like, just make sure to include the directory correctly.

%%You can also use \cite to reference bibliography items. Keep the references file in the main "diss" folder, and use \bibliography{references} in the main file to include it.  Then you can use \cite{ } (or \citep or \citet, depending on your literature format) to reference items in the bibliography.		% file with Chapter 1 contents
%\include{Chapter2/ch2}		% file with Chapter 2 contents

%\begin{thebibliography}{10}

\bibitem{baylor}
G.~I. Baylor.
Up, up and away.
{\em Proc. Roy. Soc., London A}, 294:456--475, 1959.

\bibitem{crow}
M.~E. Crow.
Aerodynamic sound emission as a singular perturbation problem.
{\em Stud. Appl. Math.}, 29:21--44, 1968.

\bibitem{dole}
Julian~D. Dole.
{\em Perturbation Methods in Applied Mathematics}.
Winsdell Publishing Company, New York, 1967.

\bibitem{fabnis}
J.~S. Fabnis, H.~J. Giblet, and H.~McDormand.
Navier-stokes analysis of solid rocket motor internal flow.
{\em J. Prop. and Power}, 2:157--164, 1980.

\bibitem{guillot}
F.~Guillot and Z.~Javalon.
Acoustic boundary layers in propellant rocket motors.
{\em J. Prop. and Power}, 5:331--339, 1989.

\bibitem{lao:thesis}
Henry Lao.
{\em Linear Acoustic Processes in Rocket Engines}.
PhD thesis, University of Colorado at Boulder, 1979.

\bibitem{lao:paper}
Q.~Lao, M.~N. Cassoy, and K.~Kirkpatrick.
Acoustically generated vorticity from internal flow.
{\em J. Fluid Mechanics}, 2:122--133, 1996.

\bibitem{lao:97}
Q.~Lao, D.~R. Kassoy, and K.~Kirkkopru.
Nonlinear acoustic processes in rocket engines.
{\em J. Fluid Mechanics}, 3:245--261, 1997.

\bibitem{mulick}
F.~C. Mulick.
Rotational axisymmetric mean flow and damping of acoustic waves in a
  solid propellant.
{\em AIAA J.}, 3:1062--1063, 1964.

\bibitem{mulick75}
F.~C. Mulick.
Stability of four-dimensional motions in a combustion chamber.
{\em Comb. Sci. Tech.}, 19:99--124, 1981.

\bibitem{richards}
R.~S. Richards and A.~M. Brown.
Coupling between acoustic velocity oscillations and solid propellant
  combustion.
{\em J. Prop. and Power}, 5:828--837, 1982.

\bibitem{smitty}
T.~M. Smitty, R.~L. Coach, and F.~B. H\"ondra.
Unsteady flow in simulated solid rocket motors.
In {\em 16st Aerospace Sciences Meeting}, number 0112 in 78. AIAA,
  1978.

\bibitem{taum}
Joseph~D. Taum.
Investigation of flow turning phenomenon.
In {\em 20th Aerospace Sciences Meeting}, number 0297 in 82. AIAA,
  1982.

\bibitem{zeddini}
Robert~A. Zeddini.
Injection-induced flows in porous-walled ducts.
{\em AIAA Journal}, 14:766--773, 1981.

\end{thebibliography}
		
%%The example dissertation has a separate .tex file for references, creating an manual bibliography environment. It is generally preferred to use BibTeX or BibLaTeX for references, which can be done by including the appropriate commands in the preamble and then using \bibliography{yourbibfile} at the end. 

\bibliographystyle{chicago}   % or apalike, plainnat, etc. — check your field's style
\bibliography{refs}     % matches the filename refs.bib

%%%%%%%%%%%%%%%%%%%%%%%%%%%%%%%%%%%%%%%%%%%%%%%%%%%%%%%%%%%%%%%%%%%
%%  Appendices

\appendix
%\include{Chapter1/app1}			% file with Appendix A contents
%%Appendices can be specific to a chapter or general to the whole dissertation, depending on how you organize your files. If it is specific, put it in the same folder as the chapter and name it app1, app2, etc. If it is general, you can put it in the same directory as the diss or in a separate folder. You can also have a mix of both.

\end{document}
